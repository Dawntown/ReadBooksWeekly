\begin{table}[htpb]
    \centering
    \caption{Some short unfamiliar terminologies}
    {\small
    \begin{tabular}{p{8em}p{26em}}
        \toprule
        Terminology & Brief definition \\
        \midrule
        \textbf{big data} vs \textbf{wide data} & $N\gg{D}$ vs $N\ll{D}$ \\
        (improve) \textbf{sample efficiency} & (reduce) the number of labeled examples needed to get good performance \\
        \textbf{language modeling} & creating unconditional generative models of text sequences (such as GPTChat)\\
        \textbf{word stemming} & replacing words with their base form\\
        \textbf{OOV} & out of vocabulary, novel terms occur in test set\\
        \textbf{mode of a dist.} & $\boldsymbol{x}^*=\mathrm{argmax}_{\boldsymbol{x}}p(\boldsymbol{x})$ \\
        \textbf{multimodal} & mode of the dist. is not unique (mode may be not a good summary)\\
        \textbf{inference} & the act of passing from sample data to generalizations, usually with calculated degrees of certainty\\
        \textbf{likelihood} & the function of parameters or the given conditions with fixed observations\\

        \bottomrule
    \end{tabular}}
    \label{tab:newtermch1}
\end{table}


\section{Introduction}
\begin{quote}
    \citep{pml1Book} -- Chapter 1.
\end{quote}

\subsection{Supervise Learning}
\textbf{Probabilistic perspective} ML: treat all unknown quantities as r.v.s.
endowed probability distribution describing a weighted set of the possible values.

\begin{note}
    Why is the probabilistic perspective adopted?
    \begin{enumerate}
        \item decision making under uncertainty and
        \item general language by other fields of science and engineering.
    \end{enumerate}
\end{note}


\textbf{Empirical risk} is the function of parameters $\boldsymbol{\theta}$, 
where the term \textit{empirical} implies using sample average to substitute population-level expectation 
(real data distribution is usually unknown), 
and the loss function $\ell(\cdot)$ can be specified according to real-world condition.
The model fitting is to find a \underline{minimizer} $\hat{\boldsymbol{\theta}}$ of $\mathcal{L}(\boldsymbol{\theta})$.
\begin{gather}
    \mathcal{L}(\boldsymbol{\theta})=\frac{1}{N}\sum_{n=1}^N
    {\ell(y_n,f(\boldsymbol{x}_n;\boldsymbol{\theta}))}
\end{gather}

\textbf{Conditional probability distribution} is adopted to capture the uncertainty.
\begin{gather}
    p(y|\boldsymbol{x};\boldsymbol{\theta})
    =f_y(\boldsymbol{x};\boldsymbol{\theta}):\mathcal{X}\to\mathcal{Y}
\end{gather}

\textbf{Deep neural networks} (vs linear\&polynomial regression) do 
nonlinear feature extraction automatically with stacked multiple hidden layers $\phi(\cdot;\boldsymbol{V})$
with $\boldsymbol{V}=[\boldsymbol{\theta}_1,\cdots,\boldsymbol{\theta}_{L-1}]$ 
and output the prediction with the final layer with $\boldsymbol{w}=\boldsymbol{\theta}_L$.
\begin{gather}
    f(\boldsymbol{x};\boldsymbol{w},\boldsymbol{V})=\boldsymbol{w}^T\phi(\boldsymbol{x};\boldsymbol{V})
\end{gather}

\begin{quote}\textit{
    Supervised learning is essentially just ``glorified curve fitting''.
}\end{quote}

\subsection{Unsupervised learning}

\textit{Unconditional} vs \textit{conditional}\unsure{
$\boldsymbol{y}$ follows mixed model with additional uncertainty of unknown co-variates
introduced by transformation from $\boldsymbol{x}$
}
models of \textit{unsupervised} vs \textit{supervised} learning.
\begin{gather}
    p(\boldsymbol{x})~\text{vs}~p(\boldsymbol{y}|\boldsymbol{x})
\end{gather}

% \begin{note}
    Strategies to find such model $p(\boldsymbol{x})$ generating data:
    \begin{enumerate}
        \item clustering by defining the similarity (distance in same space) among data points; 
        \item modeling the low-dimensional representation (Equation (\ref{eq:latent})) 
        of unobserved latent factors;
        \item self-supervised learning: 
        build proxy supervised tasks from unlabeled data, and 
        learn useful representation from the data themselves.
    \end{enumerate}
% \end{note}

\begin{example}
    \textbf{Modeling low-dimensional latent representation}\\
    \begin{gather}
        p(\boldsymbol{x}_n|\boldsymbol{z}_n;\boldsymbol{\theta})
        =N(\boldsymbol{x}_n|f(\boldsymbol{z}_n;\boldsymbol{\theta}),\boldsymbol{\Sigma})
        ~~~\boldsymbol{z}_n\in\mathbb{R}^K,~K\ll{D}
        \label{eq:latent}
    \end{gather}
    \begin{itemize}
        \item \textbf{Factor analysis}: 
        $f(\boldsymbol{z}_n;\boldsymbol{\theta})=\boldsymbol{W}\boldsymbol{z}_n+\boldsymbol{\mu}$
        \item \textbf{PCA}:
        $f(\boldsymbol{z}_n;\boldsymbol{\theta})=\boldsymbol{W}\boldsymbol{z}_n+\boldsymbol{\mu}$ and 
        $\boldsymbol{\Sigma}=\sigma^2\boldsymbol{I}$
        \item \textbf{Variational autoencoder}:
        $f(\cdot)$ neural network (nonlinear) and
        $\boldsymbol{\Sigma}=\sigma^2\boldsymbol{I}$
    \end{itemize}
\end{example}

Principles of evaluating unsupervised learning
\begin{enumerate}
    \item \textit{data compression as lossless as possible}:
    treat it as density estimation\unsure{
    recall the knowledge learned from \textit{Statistical Inference}}
    and see if a model captures the typical and useful patterns in the data;
    \item \textit{sample efficiency} of the learned representation: improvement in downstream application, 
    say, supervised learning tasks (easy to evaluate);
    \item \textit{interpretability}: discover the true underlying structure behind some dataset.
\end{enumerate}

\subsection{Reinforcement Learning}
\begin{figure}[htpb]
    \centering
    \includegraphics[width=0.7\textwidth]{figs/rl.png}
    \caption{Reinforcement Learning\protect\footnotemark[1]}
    \label{fig:rldiag}
\end{figure}
\footnotetext[1]{
\hyperlink{https://towardsdatascience.com/reinforcement-learning-101-e24b50e1d292}
{https://towardsdatascience.com/reinforcement-learning-101-e24b50e1d292}}

% \begin{quote}
%     Some key terms that describe the basic elements of an RL problem are\footnotemark[1]:
%     \begin{itemize}
%         \item Environment: Physical world in which the agent operates
%         \item State: Current situation of the agent
%         \item Reward: Feedback from the environment
%         \item Policy: Method to map agent’s state to actions
%         \item Value: Future reward that an agent would receive by taking an action in a particular state
%     \end{itemize}
% \end{quote}

Compared with supervised learning, the reward in RL may only be given occasionally and summarily
(e.g. if the agent eventually reaches a desired state after a path of actions)

\subsection{Data}

Common datasets (Benchmark):
\begin{itemize}
    \item small image datasets: MNIST [1,28,28] (extend to EMNIST, fashion-MNIST), CIFAR-10 [3,32,32];
    \item large image datasets: ImageNet [3,256,256];
    \item text datasets for text classification: IMDB movie review dataset;
    \item text datasets for machine translation: documents from multilingual organizations 
    (e.g. Canadian parliament and EU), WMT dataset;
    \item seq2seq tasks (e.g. document summarication and question answering): ?.
\end{itemize}

\textbf{TF-IDF}: term frequency-inverse document frequency.\unsure{
weighted BoW that reduce the importance of uninformative terms
}
TF$_{ij}$ is the frequency of term $i$ in document $j$, and 
IDF$_i\triangleq{\frac{N}{1+\text{DF}_i}}$, 
where $DF_i$ is the number of documents with term $i$.
\begin{gather}
    \text{TF-IDF}_{ij}=\log{\text{TF}_{ij}+1}\times\text{IDF}_i
\end{gather}

\textbf{Word embeddings}: 
\unsure{
Usually self-supervised models were adopted to build a pre-trained word embeddings 
based on large-scale contexts.
}\unsure{
The embeddings with similar meanings be often close in some metric space.}
map sparse and vocabulary-length (high-dimensional) vectors to 
dense and low-dimensional ones (embeddings).

Strategies to deal with OOV:
\begin{enumerate}
    \item replace all novel words with $\mathtt{UNK}$;
    \item byte-pair encoding: finer-grained subword structure embedding;
\end{enumerate}

Let $\boldsymbol{M}\in\mathbb{I}^{N\times{D}}$ indicates the missing status of feature $\boldsymbol{X}$:
$M_{nd}=1$ if $X_{nd}$ is missing and let $\boldsymbol{X}_v~(\boldsymbol{M}_v=\boldsymbol{0})$ and 
$\boldsymbol{X}_h~(\boldsymbol{M}_h=\boldsymbol{1})$ 
are visible and missing parts of features, respectively. 
The outcome labels $\mathbf{Y}$ are full observed.\unsure{
Missing data handling is a complex topic,
also important in \textit{clinical trial designing}, 
and will be discussed later and further.
}
\begin{itemize}
    \item \textbf{missing completely at random} (MCAR): 
    $p(\boldsymbol{M}|\boldsymbol{X}_v,\boldsymbol{X}_h,\boldsymbol{Y})=p(\boldsymbol{M})$,
    the missingness does not depend on the hidden/observed features;
    \item \textbf{missing at random} (MAR):
    $p(\boldsymbol{M}|\boldsymbol{X}_v,\boldsymbol{X}_h,\boldsymbol{Y})
    =p(\boldsymbol{M}|\boldsymbol{X}_v,\boldsymbol{Y})$,
    the missingness does not depend on the hidden features but may depend on the visible features;
    \item \textbf{not missing at random} (NMAR): otherwise
\end{itemize}

\begin{note}
    In the MCAR/MAR case, the missingness mechanism can be ignored since missingness is
    independent with the unobserved features. 
    And this book always makes the MAR assumption.
\end{note}

\section{Foundations}
\begin{quote}
    \citep{pml1Book} -- Chapter 2-8.
\end{quote}

\subsection{Probability: Univariable and Multivariable Models}

Types of uncertainty:
\begin{enumerate}
    \item \textbf{model uncertainty}: or epistemic uncertainty,
    caused by our ignorance of the underlying hidden causes or mechanism generating the data;
    \item \textbf{data uncertainty}: or aleatoric uncertainty,
    caused by intrinsic variability (the true model generating data randomly).
\end{enumerate}


\subsection{Statistics}

\subsection{Decision Theory}

\subsection{Information Theory}

\subsection{Linear Algebra}

\subsection{Optimization}
