\subsection{Trees, Forests, Bagging, and Boosting}

\begin{table}[htpb]
    \centering
    \caption{Some short unfamiliar terminologies in Tree Model}
    {\footnotesize

}
    {\small
    \begin{tabular}{cp{32em}}
        \toprule
        Terminology & Explanations \\
        \midrule
        \textbf{empirical class dist.} & $\hat{\pi}_{ic}={\#\{y_n=c:n\in\mathcal{D}_i\}}/{|\mathcal{D}_i|}$ \\
        \textbf{Gini index} & expected error rate for a node: $G_i=1-\sum_{c=1}^C\hat{\pi}_{ic}^2$ \\
        \textbf{deviance} or entropy & impurity for a node: $H_i=\mathbb{H}(\hat{\bm{\pi}}_i)=-\sum_{c=1}^C\hat{\pi}_{ic}\log\hat{\pi}_{ic}$ \\
        \textbf{surrogate split} & a highly correlated ``backup'' features used when the selected features are miss in some training or testing data. \\
        \textbf{bootstrap sampling} & sample data to the original size with \textit{replacement}, $P(\text{selected})=1-e^{-1}$, with 37\% \textbf{out-of-bag instances (oob)} \\ 
        \textbf{Gradient Boosting} & a generic paradigm of boosting methods, predicts the residual of each step (gradient) with problem specific loss, such as squared loss or regression and binary logloss for classification.\\
        \textbf{Feature importance} & the sum of the gain in accuracy for a certain feature over all non-leaf (internal) nodes \\
        \textbf{Partial dependency plots} & the impact that the most relevant features (a given feature $x_k$ at a single evaluation) have on the output \\
        \bottomrule
    \end{tabular}}
    \label{tab:treemodel}
\end{table}

Decision tree model splits the feature space into multiple region, 
the average label of the samples within which is outputted as prediction. 
The key point is the \textbf{criteria of splitting} given optimal partitioning is a NP-complete problem.
And the tree model have some key pros and cons:
\begin{itemize}
    \item [\texttt{pro}] insensitive to monotone transformations of the inputs since ranking is used for partitioning,
    so need to standardize the data\unsure{
    but the distribution of training and testing data still should be similar or same.
    };
    \item [\texttt{pro}] with automatic variable selection;
    \item [\texttt{pro}] robust to outliers and able to handle missing data.
    \item [\texttt{con}] unable to predict very accurately;
    \item [\texttt{con}] whose structures are unstable which small changes to the inputs can have large effect on 
    due to the hierarchical nature of tree growing process. (\textbf{high variance predictor})
\end{itemize}



\textbf{Ensemble learning} comprehensively integrates the predictions from multiple models
\begin{itemize}
    \item \textit{votting} or committee method: simply average the predictions for regression or take the majority class for classification, with the same bias with the base models but less variance and higher accuracy especially when $M=|\mathcal{M}|$ is large 
    \begin{gather}
        p=P(\sum_{m=1}^M Y_m>\frac{M}{2})=1-F_{\text{binom}(M,\theta)}(\frac{M}{2})
    \end{gather}
    
    \item \textbf{Stacking}, weighted voting, learn the combination weights on a \textit{separate} dataset.\unsure{Comparing with Bayes model averaging: 
    $p(y|\bm{x},\mathcal{D})=\sum_{m\in\mathcal{M}}p(m|\mathcal{D})p(y|\bm{x},m,\mathcal{D})$
    }
    \begin{gather}
        f(y|\bm{x}) = \sum_{m\in\mathcal{M}}w_mf_m(y|\bm{x})
    \end{gather}
    
    \item \textbf{Bagging}, i.e. \textbf{bootstrap aggregating}, train on the selected data 
    and validate on the oob, then (weightedly) average the prediction from models.
    \textbf{Random Forest} is Bagged DT \textit{additionally} based on a randomly chosen subset of features\unsure{
    ``Randomness'' of each tree in the ``forest'' comes from randomly selected samples of bagging and randomly chosen features.
    } to decorrelate the base learners.
    However, bagging only uses 63\% of data for training, which usually damages the performance of DNN.\unsure{
    The base learner can be regarded as \textbf{adaptive basis function} in the linear model.
    }

    \item \textbf{Boosting} sequentially fitting additive models (strong learner): training a base classifier (weak learner) with the \textit{weights of data samples by the errors} made by the previous ones. At each step, the parameters to be optimized include base learner parameters $\bm{\theta}_m$ and the weight of new base learner $\theta_m$ 
    \begin{gather}
        (\bm{\theta}_m.\beta_m)=\argmin_{\bm{\theta}, \beta}\sum_{i=1}^{N}\ell(y_i,f_{m-1}(\bm{x}_i)+\beta F(\bm{x}_i;\bm{\theta}))
    \end{gather}
    \begin{itemize}
        \item \textbf{least squares boosting}: $\ell(y,\hat{y}) = (y-\hat{y})^2$\unsure{
        for regression problem
        }
        \begin{gather}
            \ell(y_i,f_{m-1}(\bm{x}_i)+\beta F(\bm{x}_i;\bm{\theta})) = (r_{im}-\beta F(\bm{x}_i;\bm{\theta}))^2
        \end{gather}

        \item \textbf{exponential loss and AdaBoost}: $\ell(\Tilde{y},\hat{y})=\exp(-\Tilde{y}\hat{y})$\unsure{
        for classification problem. It is also a smooth upper bound on the 0-1 loss. \textbf{\color{red}NOTE} that the exp loss can be written like this only when the label space is $\Tilde{y}\in\{-1,1\}$.
        }\unsure{
        Sensitive to outliers.
        }
        \begin{example}[AdaBoost]
            At step $m$, the optimization objective is 
            \begin{align}
                &~\min_{F,\beta} L_m(F;\beta) \\
                L_m(F,\beta) 
                =&~ \sum_{i=1}^N\exp\left\{
                    -\Tilde{y}_i\left[
                        f_{m-1}(\bm{x}_i) + \beta F(\bm{x}_i)
                    \right]
                \right\} \\
                =&~ \sum_{i=1}^N
                \underbrace{\exp\left\{-\Tilde{y}_if_{m-1}(\bm{x}_i)\right\}}_{\omega_{im}~\text{weight}}
                \exp\left\{
                    -\beta\Tilde{y}_i F(\bm{x}_i)
                \right\} \\
                % \overset{F,\Tilde{y}\in\{-1,1\}}
                {=}&~ (e^\beta - e^{-\beta})\sum_{i=1}^N\omega_{im}\mathbb{I}(\Tilde{y}_i\neq F(\bm{x}_i)) + e^{-\beta}\sum_{i=1}^N\omega_{im} \\
                \Rightarrow
                % \begin{cases}
                F_m =&~ \argmin_{F}\sum_{i=1}^N\omega_{im}\mathbb{I}(\Tilde{y}_i\neq F(\bm{x}_i))\\
                \beta_m =&~ \frac{1}{2}\log\frac{1-\text{err}_m}{\text{err}_m} \\
                % \end{cases}
                \text{where err}_m =&~ \frac{\sum_{i=1}^N\omega_{im}\mathbb{I}(\Tilde{y}_i\neq F_m(\bm{x}_i))}{\sum_{i=1}^N\omega_{im}}
            \end{align}
        \end{example}
        \bigskip
        \item \textbf{log loss and LogitBoost}: $\ell(\Tilde{y},\hat{y})=1/(1+\exp(-2\Tilde{y}\hat{y}))$
        \item $\cdots$
    \end{itemize}
\end{itemize}



% \textbf{Gradient Boosting}: the idea is similar with the above boosting methods but with general losses and solving the $\hat{\bm{f}}=\argmin_{\bm{f}}\mathcal{L}(\bm{f})$ by performing gradient descent in the space of functions

There are a few more improvement in \textbf{XGBoost} compared with normal gradient tree boost:
\begin{enumerate}[(1)]
    \item add regularization term in the loss function,.
    \begin{gather}
        \Omega(f)=\gamma J + \frac{1}{2}\sum_{j=1}^Jw_j^2,
    \end{gather}
    where $J$ is the number of leaves, and the $w_j$ is the leaf's output.
    
    \item apply 2-nd order optimization on the loss,\unsure{
    In the context of regression tree, $F(\bm(x))=w_{q(\bm{x})}$, where $q:\mathbb{R}^D\to \{1,\cdots,J\}$.
    \color{red}However, how to calculate the gradient and hessian of loss wrt the function space of model?
    }
    {\begin{align}
        \mathcal{L}_m(F_m)
        \approx&~ \sum_{i=1}^N\left[
            \ell(y_i,f_{m-1}(\bm{x}_i)) + g_{im}F_m(\bm{x}_i) + \frac{1}{2}h_{im}F_m^2(\bm{x}_i)
        \right] \\
        &~+ \Omega(F_m) + \text{const}
    \end{align}}
    % \begin{gather}
    %     g_{im} = \left.
    %     \frac{\partial \ell(y_i,f(\bm{x}_i))}{\partial f(\bm{x}_i)}
    %     \right|_{f=f_{m-1}},~~
    %     h_{im} = \left.
    %     \frac{\partial^2 \ell(y_i,f(\bm{x}_i))}{\partial f(\bm{x}_i)^2}
    %     \right|_{f=f_{m-1}}
    % \end{gather}
    where 
    $g_{im} = \left.
    \frac{\partial \ell(y_i,f(\bm{x}_i))}{\partial f(\bm{x}_i)}
    \right|_{f=f_{m-1}}$, and 
    $h_{im} = \left.
    \frac{\partial^2 \ell(y_i,f(\bm{x}_i))}{\partial f^2(\bm{x}_i)}
    \right|_{f=f_{m-1}}$.
\end{enumerate}

\section{Beyond Supervised Learning}


\begin{table}[htpb]
    \centering
    \caption{Some short unfamiliar terminologies beyond Supervised Learning}
    {\footnotesize

}
    {\small
    \begin{tabular}{cp{32em}}
        \toprule
        Terminology & Explanations \\
        \midrule
        \textbf{AutoAugment} & Using blackbox optimization methods to learn which augmentations work best. \\
        \textbf{adapters} & keep the pre-trained model untouched, but add new parameters (layers inserted inside of the pre-trained model) to modify its internal behavior to customize the feature extraction process for each task \\
        \textbf{Imputation task} & aka fill-in-the blank task or cloze task (in NLP), $\hat{\bm{x}}_h=f(\bm{x}_v,\bm{x}_h=\mathbf{0})$ \\
        \textbf{proxy task} & aka pretext task, $p(y|\bm{x}_1,\bm{x}_2)=p(y|r(f(\bm{x}_1),f(\bm{x}_2)))$ to predict the relationship (proxy label that we \textit{set}) between $\bm{x}_1$ and $\bm{x}_2$ by a common representation learning \\
        \textbf{\makecell{conditional energy\\based model}} & model with the form $p(\bm{x}_2|\bm{x}_1)={\exp\{-\mathcal{E}(\bm{x}_2|\bm{x}_1)\}}/{Z(\bm{x}_1)}$, $Z(\bm{x})=\int{\exp\{-\mathcal{E}(\bm{y}|\bm{x})\}}d\bm{y}$ (\textbf{partition function}, a normalization constant) \\ 
        \textbf{MoCo} & momentum contrastive learning, when there is no enough diverse negatives in a batch, a memory bank is built for generating new negative embeddings by exponential moving averaging \\
        \textbf{promot engineering} & disambiguating phrases added hand when the raw class names can be ambiguous across/within a dataset \\
        \textbf{domain adaptation} & inputs from different domain (source and target domains), $\mathcal{X}_s\neq\mathcal{X}_t$, but outputs from a common domain, $\mathcal{Y}$, 
        \textit{dual} of \textbf{transfer learning}, which has inputs from a common domain and outputs from different domains. \\
        \textbf{pseudo labeling} & aka self-training, use the predictions from the model itself on unlabeled data (by \textit{offline} or \textit{online}) as ``labels'' for subsequent training. \\
        \textbf{confirmation bias} & the model is continually confirming its own (incorrect) bias about the decision rule if the model generates incorrect predictions for unlabeled data and retrains on them.\\
        \textbf{cluster assumption} & the decision boundary between classes should fall in a low-density region of the data manifold which can be estimated from the unlabeled data and derives \textit{entropy minimization}. \\
        \textbf{manifold assumption} & Two datapoints ``similar'' in some meaningful way are expected to share a label. \\
        \textbf{transductive learning} & learn to predict labels for a \textbf{fixed} unlabeled dataset, rather than learn a model that generalizes labels for any new input, which is called \textbf{inductive learning}. \\
        \textbf{meta learning} & learning the parameters of the estimator itself ($\hat{f}(\bm{x};\bm{\theta}(\mathcal{D}_{j+1};{\color{red}\bm{\phi}(\mathcal{D}_{1:J})}))$, learning to learn). \\
        \textbf{MAML} & model-agnostic meta-learning, $\bm{\phi}^*=\argmax_{\bm{\phi}}\frac{1}{J}\sum_{j=1}^J\log p(\mathcal{D}_\text{val}^j|\hat{\theta}_{j}(\bm{\phi},\mathcal{D}_\text{tr}^j))$, equivalent to an \textit{approximate MAP estimate} using a Gaussian prior centered at $\bm{\phi}$. \\
        \textbf{Few-shot learning} & learn to predict fro very few labeled examples, 
        in \textit{extreme} situations, the ``few'' can reduce to one or zero. \\
        \textbf{C-way N-shot classification} & a common way to evaluate FSL methods, classifying $C$ classes using only $N$ examples for \textit{each} class.\\
        \textbf{Matching networks} & learn a metric measuring the similarity on some \textit{external} dataset, and then compare the new input $\bm{x}$ to all the labeled examples $\bm{x}_n$ in the context. \\
        \textbf{weakly supervised learning} & the labels $y$ in training set are not exact (a general term) for classification. \\
        \textbf{label smoothing} & A distribution, $p(y|\bm{x}_n)$, compared with the original delta function, is applied for each label in the cross entropy loss $\mathcal{L}(\bm{\theta})=-\sum_n\sum_y {\color{blue}p(y|\bm{x}_n)}\log q_{\bm{\theta}}(y|\bm{x}_n)$ \\
        \textbf{multi-instance learning} & only labeling a bag/group of examples \\
        \textbf{distance supervision} & use some known facts to label other unlabled data based on the other data's \textit{semantic relationship} (I guess) with the fact. \\
        \bottomrule
    \end{tabular}}
    \label{tab:fewexamplars}
\end{table}

\subsection{Learning with Fewer Labeled Examples}

\textbf{Theoretical justification} for \textbf{data augmentation}: a way to algorithmically inject prior knowledge
\begin{align}
    &~p_\mathcal{D}(\bm{x},\bm{y}) = \frac{1}{N}\sum_{n=1}^N\delta(\bm{x}-\bm{x}_n)\delta(\bm{y}-\bm{y}_n)\\
    \to
    &~p_\mathcal{D}(\bm{x},\bm{y}|A) = \frac{1}{N}\sum_{n=1}^N p(\bm{x}|\bm{x}_n,A)\delta(\bm{y}-\bm{y}_n)\\
    \text{e.g.}&~~p(\bm{x}|\bm{x}_n,A)=\mathcal{N}(\bm{x}|\bm{x}_n,\sigma^2\mathbf{I})
\end{align}


 \textbf{Contrastive task} creates pairs of examples semantically similar to each other (e.g. using data augmentation methods),
 and then to ensure that the distance between their representations is closer than the distance between unrelated examples. 
 For examples,
\begin{enumerate}[(1)]
    \item \textbf{SimCLR}: \textit{Simple contrastive learning of visual representations},\unsure{
    consistent with the basic concept of contrastive learning
    }
    maximize the similarity (cosine similarity) between the views augmented from the same sample and minimize that between unrelated views from different samples.
    \item \textbf{CLIP}: \textit{Contrastive Language-Image Pre-training}, determine if one strings is more likely to be the correct text paired with the given image than another string,\unsure{
    the idea of contrastivity is similar with SimCLR but the counterparts to paired are image and text, respectively, instead of augmented images from one, 
    so the normalization of embedding is required considering heterogeneous embedding spaces are involved.
    }
    which can be used to \textbf{zero-shot classification}.
\end{enumerate}

\textbf{Domain adversarial learning} minimizes the loss of classifying label $y$ but maximize the loss of classifying the source domain $d$
{\small\begin{gather}
    \min_{\bm{\phi}}\max_{\bm{\theta}} \frac{1}{|\mathcal{D}_s|+|\mathcal{D}_t|} \sum_{n\in\mathcal{D}_s,\mathcal{D}_t}
    \ell(d_n,\bm{f}(\bm{x}_n;\bm{\theta})) + \frac{1}{|\mathcal{D}_s|}\sum_{m\in\mathcal{D}_s}\ell(y_m,g(\bm{f}(\bm{x}_m;\bm{\theta});\bm{\phi})),
\end{gather}}
where $\bm{f}(\cdot;\bm{\theta}):\mathcal{X}_s\cup\mathcal{X}_t\to\mathcal{H}$, and $g(\cdot;\bm{\phi}):\mathcal{H}\to\mathcal{Y}$.
This can be implemented by the \textit{gradient sign reversal} trick.


\textbf{Input-output mutual information}: maximize the mutual information of input and output and can also derive \textit{entropy minimization}\unsure{
Another principle is \textit{cluster assumption}
} 
but under a priori that all classes are equally likely. 
\begin{gather}
    \mathcal{I}(y;\bm{x})=\underbrace{\mathbb{E}_{\bm{x}}\sum_{i=1}^Lp(y_i|\bm{x})\log p(y_i|\bm{x})}_\text{neg. \textit{entropy minimization} obj.}
        -\underbrace{\sum_{i=1}^L\mathbb{E}_{\bm{x}}[p(y_i|\bm{x})\log\mathbb{E}_{\bm{x}}p(y_i|\bm{x})]}_\text{equal class proba. assumption}
\end{gather}

\textbf{Co-Training}, similar to self-training, additionally assumes that there are 
\textit{two complementary} (informative-but-independent) ``views'' of the data, 
both of which can be used separately to train a reasonable model, and for a given new input, 
only high-confident pseudo label is used for retraining.

\textbf{Tri-Training} trains three models based on independently sampled datasets with replacement, 
where the pseudo-labels output agree on two will be feeded to retrain the third, and repreat the process iteratively.

\textbf{Label propagation on graphs}, according to the manifold assumption, propagates the known labels across
edges of the graph in such a way that there is minimal disagreement in the labels of a given node's neighbors
according to the similarity between samples.
\begin{align}
    \mathbf{Y}\leftarrow&~\mathbf{TY}\\
    Y_{ic}\leftarrow&~\frac{Y_{ic}}{\sum_{k}Y_{ik}}
\end{align}
where $\mathbf{T}\in\mathbf{R}^{(M+N)\times(M+N)}$ is transition matrix defined by some measure of similarity between sample $i$ and $j$, $w_{ij}$ that $T_{ij}=\frac{w_{ij}}{\sum_{k}w_{ij}}$, and
$\mathbf{Y}\in\mathbb{R}^{(M+N)\times C}$ is label matrix representing the probability of classes for each sample.

\textbf{Consistency regularization}: the perturbing on a given datapoint should not cause the model's output to change \textbf{dramatically}.
\begin{gather}
    \mathcal{L}(\bm{\theta}) = -\sum_{i=1}^M\log p(y=y_i|\bm{x}_i;\bm{\theta}) + \lambda\sum_{j=1}^N\underbrace{D(p(y|\bm{x}_j;\bm{\theta}),p(y|\bm{x}'_j;\bm{\theta}))}_{\text{output diff. of }\bm{x}~\&~\bm{x}'\sim q(\bm{x}'|\bm{x})}
\end{gather}\unsure{
The difference measurement $D(\cdot,\cdot)$ can be L2 distance or KL divergence, etc
}
The $\lambda$ takes value from 0 and increases over the cource of training since the prediction is bad during the beginning.
The transformation $q(\bm{x}'|\bm{x})$ can be strong data augmentations that heavily corrupt the input but still do not change the label,
which requires the expert knowledge or learn a $\bm{\delta}$ of $\bm{x}$, called \textbf{virtual adversarial training (VAT)}, such that
\begin{align}
    \bm{\delta}
    =&~\argmax_{\bm{\delta}}\infdiv{p(y|\bm{x};\bm{\theta})}{p(y|\bm{x}+\bm{\delta};\bm{\theta})} \\
    \approx
    \bm{\delta}
    \leftarrow&~\left.\nabla_{\bm{\delta}}\infdiv{p(y|\bm{x};\bm{\theta})}{p(y|\bm{x}+\bm{\delta};\bm{\theta})}\right|_{\bm{\delta=\xi\bm{d}}} \\
    \text{initializing}&~\bm{d}\sim\mathcal{N}\\
    \Rightarrow \bm{x}'
    =&~\bm{x}+\frac{\epsilon}{\|\bm{\delta}\|_2}\bm{\delta}
\end{align}

\textbf{VAE based semi-supervised learning} use VAE as the backbone to transform $\bm{x}$ from a latent variable $\bm{z}$ 
whose dimensions are independent and lower than that of original, \textit{without} (\textbf{M1})\unsure{
M1 is the standard VAE to get a low dimension representation of $\bm{x}$ without the labels.
} or \textit{with} (\textbf{M2}) label $y$ like Figure \ref{fig:vaem1m2}.\unsure{
To understand the following deduction, refer to the EM algorithm and ELBO in Chapter 8 Optimization.
}.
\begin{align}
    p_{\bm{\theta}}(\bm{x},y)
    &= p_{\bm{\theta}}(y)p_{\bm{\theta}}(\bm{x}|y) = p_{\bm{\theta}}(y)\int p_{\bm{\theta}}(\bm{x}|y,\bm{z})p_{\bm{\theta}}(\bm{z})d\bm{z}~~~\text{(labeled data)} \\
    \text{with}~&\begin{cases}
        p_{\bm{\theta}}(\bm{z}) = \mathcal{N}(\bm{z}|\overbrace{\bm{\mu}_{\bm{\theta}}, \mathbf{\Sigma}_{\bm{\theta}}}^{\text{usually standard}~\mathcal{N}}) & \text{latent prior} \\
        p_{\bm{\theta}}(y) = \text{Cat}(y|\bm{\pi}_{\bm{\theta}}) & \text{label prior} \\
        p_{\bm{\theta}}(\bm{x}|y,\bm{z}) = p(\bm{x}|\bm{f}_{\bm{\theta}}(y,\bm{z})) & \text{likelihood}
    \end{cases} \\
    {\color{purple}q_{\bm{\phi}}(\bm{z}|y,\bm{x})} 
    &{\color{purple}= \mathcal{N}(\bm{z}|\mu_{\bm{\phi}}(y,\bm{x}),\mathrm{diag}(\sigma^2_{\bm{\phi}}(\bm{x})))} \label{eq:m2confuse}\\
    \log p_{\bm{\theta}}(\bm{x},y) 
    &\geq \mathbb{E}_{q_{\bm{\phi}}(\bm{z}|\bm{x},y)}\underbrace{\left[
        \log p_{\bm{\theta}}(\bm{x}|y,\bm{z}) + \log p_{\bm{\theta}}(y) + \log p_{\bm{\theta}}(\bm{z}) - \log q_{\bm{\phi}}(\bm{z}|\bm{x},y)
    \right]}_{{p_{\bm{\theta}}(\bm{x},y,\bm{z})}/{q_{\bm{\phi}}(\bm{z}|\bm{x},y)}}\\
    &=:-{\color{purple}L(\bm{x},y;\bm{\theta},\bm{\phi})} \\
    p_{\bm{\theta}}(\bm{x})
    &=  \int p_{\bm{\theta}}(y)\int p_{\bm{\theta}}(\bm{x}|y,\bm{z})p_{\bm{\theta}}(\bm{z})d\bm{z}dy~~~\text{unlabeled data} \\
    {\color{teal}q_{\bm{\phi}}(\bm{z},y|\bm{x})} 
    &{\color{teal}=q_{\bm{\phi}}(\bm{z}|\bm{x})q_{\bm{\phi}}(y|\bm{x})} \\ 
    {\color{teal}q_{\bm{\phi}}(\bm{z}|\bm{x})}
    &{\color{teal} = \mathcal{N}(\bm{z}|\bm{\mu}_{\bm{\phi}}(\bm{x}),\mathrm{diag}(\sigma^2_{\bm{\phi}}(\bm{x})))} \\
    {\color{blue}q_{\bm{\phi}}(y|\bm{x})}
    &{\color{blue}=\boxed{\mathrm{Cat}(y|\bm{\pi}_{\bm{\phi}}(\bm{x}))} }~~~\text{classifier for unlabeled data}\\
    \log p_{\bm{\theta}}
    &\geq \mathbb{E}_{q_{\bm{\phi}}(\bm{z},y|\bm{x})}\underbrace{\left[
        \log p_{\bm{\theta}}(\bm{x}|y,\bm{z}) + \log p_{\bm{\theta}}(y) + \log p_{\bm{\theta}}(\bm{z}) - \log q_{\bm{\phi}}(\bm{z},y|\bm{x})
    \right]}_{p_{\bm{\theta}}(\bm{x},y,\bm{z})/q_{\bm{\phi}}(\bm{z},y|\bm{x})} \\
    &= -\sum_{y} q_{\bm{\phi}}(y|\bm{x}){\color{purple}L(\bm{x},y;\bm{\theta},\bm{\phi})} + \mathbb{H}(q_{\bm{\phi}}(y|\bm{x})) \\
    &=: -{\color{teal}U(\bm{x};\bm{\theta},\bm{\phi})}\\
    \Rightarrow \mathcal{L}(\bm{\theta},\bm{\phi}) 
    &= \mathbb{E}_{(\bm{x},y)\sim\mathcal{D}_L}{\color{purple}L(\bm{x},y;\bm{\theta},\bm{\phi})}\\
    &~~~+ \mathbb{E}_{\bm{x}\sim\mathcal{D}_U}{\color{teal}U(\bm{x};\bm{\theta},\bm{\phi})} \\
    &~~~+\alpha \mathbb{E}_{(\bm{x},y)\sim\mathcal{D}_L}[-\log {\color{blue}q_{\bm{\phi}}(y|\bm{x})}]
\end{align}\unsure{
The notations here seems to be confusing, needed to be verified.
{\color{red}This part is still not very understood... And the notation used here may be incorrect.}
}

\begin{figure}[hptb]
    \centering
    \parbox{0.7\textwidth}{
    \begin{framed}
    \parbox{.2\textwidth}{%
        \subcaptionbox{M1\label{fig:vaem1}}{\centering
        \begin{tikzpicture}[
        roundnode/.style={circle, draw=violet!60, fill=violet!5, minimum size=7mm},
        ->,>=stealth',auto,node distance=3.5em,thick]
        \tikzstyle{every node}=[font=\small,scale=0.9]
            \node[roundnode] (data) {$\bm{x}$};
            \node[roundnode, right = of data] (latent) {$\bm{z}$};
              
            \draw[->] (latent)  -- (data);
        \end{tikzpicture}
        }}
    \hskip5em
    \parbox{.35\textwidth}{%
        \subcaptionbox{M2\label{fig:vaem2}}{
        \begin{tikzpicture}[
        roundnode/.style={circle, draw=violet!60, fill=violet!5, minimum size=7mm},
        ->,>=stealth',auto,node distance=3.5em,thick]
        \tikzstyle{every node}=[font=\small,scale=0.9]
            \node[roundnode] (data) {$\bm{x}$};
            \node[roundnode, left = of data] (label) {$y$};
            \node[roundnode, right = of data] (latent) {$\bm{z}$};
              
            \draw[->] (latent) -- (data);
            \draw[->] (label) -- (data);
        \end{tikzpicture}
        }}
    \end{framed}
    }
    \caption{Two structures of VAE model in semi-supervised learning}
    \label{fig:vaem1m2}
\end{figure}


\textbf{Active learning} identify the true predictive mapping $y=f(\bm{x};\bm{\theta})$ by querying \textit{as few points as possible} such as in \textit{Bayesian optimization} and \textit{experiment design}.\unsure{
highlight choosing strategy of the labeled data points
}
\begin{itemize}
    \item \textbf{decision-theoretic approach}: explore a new point $\bm{x}$ with the minimal average \textit{risk increment} over all kind of output $y$ and parameters $\bm{\theta}$.
    \begin{align}
        U(\bm{x})\triangleq&~\mathbb{E}_{p(y|\bm{x}|\mathcal{D})}\left[
            \min_{a}(R(a|\mathcal{D})-R(a|\mathcal{D},(\bm{x},y)))
        \right]\\
        R(a|\mathcal{D}) =&~\mathbb{E}_{p(\bm{\theta}|\mathcal{D})}\ell(\bm{\theta},a)
    \end{align}

    \item \textbf{information-theoretic approach}: similar with the decision-theoretic one but considering to maximize \textit{information gain} about the parameters $\bm{\theta}$
    \begin{align}
        U(\bm{x}) =&~ \mathbb{H}(p(\bm{\theta}|\mathcal{D})) - \mathbb{E}_{p(y|\bm{x},\mathcal{D})}\mathbb{H}(p(\bm{\theta}|\mathcal{D},(\bm{x},y))) \\
        =&~ \mathbb{I}(\bm{\theta},y|\mathcal{D},\bm{x}) \\
        =&~ \underbrace{\mathbb{H}(p(y|\bm{x},\mathcal{D}))}_\text{\makecell{max. entropy\\ sampling}} - \underbrace{\mathbb{E}_{p(\bm{\theta}|\mathcal{D})}\mathbb{H}(p(y|\bm{x},\bm{\theta}))}_\text{\makecell{pick example with\\confident prediction}} \label{eq:actlearninfo}
    \end{align}
    which will select examples $\bm{x}$ making confident predictions which are highly diverse. 
    Therefore this approach is called \textbf{Bayesian active learning by disagreement (BALD)}.

    \item \textbf{batch active learning} or \textbf{BatchBALD}\unsure{
    The above strategies are greedy (myopic) search, select a optimal sample at one time.
    } collect a set of $B$ samples $(\mathbf{X}, \mathbf{Y})$ by a sequential greedy search, which is within a \textit{constant factor} of optimal.
\end{itemize}

 

