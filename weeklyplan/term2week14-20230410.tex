\begin{table}[htpb]
    \centering
    \caption{Some short unfamiliar terminologies in Remaining chapter}
    {\footnotesize

}
    {\small
    \begin{tabular}{cp{32em}}
        \toprule
        Terminology & Explanations \\
        \midrule
        \textbf{UMAP} & Uniform Manifold Approximation and Projection, similar to tSNE at a high level 
        but tends to \textit{preserve global structure} better and is much faster. \\
        \textbf{\makecell[t]{term-document\\frequency matrix}} & $\mathbf{C}\in\mathbb{R}^{M\times N}$, $C_{ij}$ be the number of times \textit{term} $i$ occurs in \textit{context} $j$, which is a simplified TF-IDF. \\
        \textbf{\makecell[t]{latent semantic\\indexing}} & $\mathbf{C}\approx\hat{\mathbf{C}}=\mathbf{USV}^\mathsf{T}$, $\mathbf{U}\in\mathbb{R}^{M\times L}$ are the embeddings for $M$ \textit{words}, and 
        $\mathbf{VS}=[\bm{v}_j\odot\bm{s}]_{j=1:N}\in\mathbb{R}^{N\times L}$ are the embeddings for $N$ \textit{documents}. \\
        \textbf{\makecell[t]{latent semantic\\analysis}} & same with LSI but with a finer grained $\mathbf{C}$ defined on the \textit{context of window $h$}, representing the \textbf{co-occurring} probability of maximum $h$ words. \\
        \textbf{\makecell[t]{pointwise mutual\\information}} & $C_{ij}=\mathrm{PMI}(i,j)\triangleq\log\frac{p(i,j)}{p(i)p(j)}$ or $[\mathrm{PMI}(i,j)]_+$, another substitution of term-document frequency matrix for the latent semantic preprocessing above \\
        \textbf{GLOVE} & global vectors for word representation \\
        \textbf{Word analogies} & ``man is to woman as king is to queen.'' \\
        \textbf{Rand-walk model} & $p(w_t=w|\bm{z}_t)=\exp(\bm{z}_t^\mathsf{T}\bm{v}_w)/\underbrace{\sum_{w'}\exp(\bm{z}_t^\mathsf{T}\bm{v}_{w'})}_{Z(\bm{z}_t)}$, assuming the prior for the word embeddings $\bm{v}_t$ is an isotropic Gaussian and the latent topic $\bm{z}_t$ undergoes a slow Gaussian random work, which can interpret the word embedding methods. \\
        \bottomrule
    \end{tabular}}
    \label{tab:rest}
\end{table}



\textbf{Stochastic neighborhood embedding (SNE)} converts high-dimensional Euclidean distances, 
including original data space and embedding space, into conditional probabilities that represent similarities.
\unsure{
    This idea is similar with MDS but uses conditional probability of picking datapoint $j$ given datapoint $i$
    as the similarity measure.
}
\begin{gather}
    \begin{cases}
        p_{j|i} = \frac{\exp\left(-\frac{1}{2\sigma_i^2}\|\bm{x}_i-\bm{x}_j\|^2_2\right)}{\sum_{k\neq i}\exp\left(-\frac{1}{2\sigma_i^2}\|\bm{x}_i-\bm{x}_k\|^2_2\right)} \\
        q_{j|i} = \frac{\exp\left(-\|\bm{z}_i-\bm{z}_j\|^2_2\right)}{\sum_{k\neq i}\exp\left(-\|\bm{z}_i-\bm{z}_k\|^2_2\right)}
    \end{cases}\\
    \mathcal{L}(\mathbf{Z}) = \sum_{i}\infdiv{P_i}{Q_i} = \sum_i\sum_j p_{j|i}\log\frac{p_{j|i}}{q_{j|i}} \\
    \nabla_{\bm{z}_i}\mathcal{L}(\mathbf{Z}) = 2\sum_{j}(\bm{z}_j-\bm{z}_i)(p_{j|i}-q_{j|i}+p_{i|j}-q_{i|j})
\end{gather}
Intuitively, $q_{j|i}$ should match $p_{j|i}$ if the embedding is good. 
The points are pulled towards each other if the $p$'s are bigger than the $q$'s, 
and repelled if the $q$'s are bigger than the $p$'s.
However, the objective is not convex and needs SGD for optimization.
\begin{itemize}
    \item \textbf{Symmetric SNE} simplifies the conditional probability to a symmetric one and has similar result with classic SNE.
    \begin{gather}
        \begin{cases}
            p_{ij} = \frac{\exp\left(-\frac{1}{2\sigma^2}\|\bm{x}_i-\bm{x}_j\|^2_2\right)}{\sum_{k<l}\exp\left(-\frac{1}{2\sigma^2}\|\bm{x}_k-\bm{x}_l\|^2_2\right)} \\
            q_{ij} = \frac{\exp\left(-\|\bm{z}_i-\bm{z}_j\|^2_2\right)}{\sum_{k<l}\exp\left(-\|\bm{z}_k-\bm{z}_l\|^2_2\right)}
        \end{cases}\\
        \mathcal{L}(\mathbf{Z}) = \sum_{i}\infdiv{P}{Q} = \sum_{i<j} p_{ij}\log\frac{p_{ij}}{q_{ij}} \\
        \nabla_{\bm{z}_i}\mathcal{L}(\mathbf{Z}) = 2\sum_{j}(\bm{z}_j-\bm{z}_i)(p_{ij}-q_{ij})
    \end{gather}

    \item \textbf{t-distributed SNE}, comparing with \textit{symmetric SNE}, 
    uses heavy tail distribution ($\nu=1$ Student-t, i.e. Cauchy) to calculate probability in embedding space
    to solve the \textit{crowding problem} brought by Gaussian distribution, 
    which tends to squeeze points that are relatively far away in the high dimensional space close together in the low dimensional embedding space\unsure{
        \color{red}Why this problem can arise due to non-heavy tail distribution?
    }
    \begin{gather}
        q_{ij}=\frac{\left(1+\|\bm{z}_i-\bm{z}_j\|_2^2\right)^{-1}}{\sum_{k<l}\left(1+\|\bm{z}_k-\bm{z}_l\|_2^2\right)^{-1}} \\
        \nabla_{\bm{z}_i}\mathcal{L}(\mathbf{Z}) = 4\sum_{j}(p_{ij}-q_{ij})\underbrace{(\bm{z}_i-\bm{z}_j){\color{blue}(1+\|\bm{z}_i-\bm{z}_j\|^2)^{-1}}}_{\bm{w}}
    \end{gather}
    where $(1+\|\bm{z}_i-\bm{z}_j\|^2)^{-1}$ lets points in embedding space act like stars and galaxies, 
    forming many \textbf{well-separated} clusters (galaxies) each of which has many stars tightly packed inside.\unsure{
        From the gradient, we can see the distribution difference of $p_{ij}$ and $q_{ij}$ only takes an effect within a local range and slowly decays with the distance increasing,
        which means only close points affect learning of congregation, but the distance between remote points will not or take few effect. 
        The bandwidth $\sigma^2$ controls the distance of neighbors such that is close enough to affect the learning.
    }
\end{itemize}

\begin{figure}[htpb]
    \centering
    \includestandalone[width=.8\textwidth]{figs/cauchypdf_raw}
    \caption{Plots of $x(1-x^2)^{-1}$ and $(1-x^2)^{-1}$}
    \label{fig:cauchypdf}
\end{figure}

\textbf{Word2vec} are \textit{shallow} neural nets for predicting a word given its context.
\begin{itemize}
    \item \textbf{CBOW}: continuous bag of words. For a seq of words $\bm{w}$ of length $T$, its log-likelihood is defined as follow, i.e., \uline{each word is predicted from its context}.\unsure{
        Every word depends on its neighbors within the distance of radius $m$.
    }\unsure{
        The formula here seems incorrect.
    }
    \begin{align}
        \log p(\bm{w}) & = \sum_{t=1}^T\log p(w_t|\bm{w}_{(t-m):(t+m)})=\sum_{t=1}^T\log\frac{\exp(\bm{v}_{w_t}^\mathsf{T}\bar{\bm{v}}_{\color{red}t})}{\sum_{t'}\exp(\bm{v}_{w_{t'}}^\mathsf{T}\Bar{\bm{v}}_{\color{red}t})} \\
        \bar{\bm{v}}_{\color{red}t} & \triangleq \frac{1}{2m}\sum_{h=1}^m(\bm{v}_{w_{t-h}}+\bm{v}_{w_{t+h}})
    \end{align}

    \item \textbf{Skip-gram} \uline{predict the context given each word}
    \begin{align}
        \log p(\bm{w}) & = \sum_{t=1}^T\sum_{-m\leq j\leq m}\log p(w_{t+j}|w_t) \\
        \log p(w_o|w_c) & = \bm{v}_{w_o}^\mathsf{T}\bm{v}_{w_c} - \log\sum_{i\in\mathcal{V}}\exp(\bm{v}_{w_i}^\mathsf{T}\bm{v}_{w_c})
    \end{align}
    Since the vanilla skip-gram should compute the preactivations for the entire vocabulary, computationally intensive,
    a set of $K+1$ \textit{context words} is sampled for each central word, $w_t$, to substitute the vocabulary related term, 
    called \textbf{Negative sampling}.
    In the sampled \textit{context words}, only one term is indeed in context of central words, $w_{t+j}$, $j=-m,\cdots,m$,
    and the remaining $K$ words, $\mathcal{K}$, are sampled from vocabulary according to a reweighed unigram distribution, 
    $w_k\sim p(w)\propto\mathrm{freq}(w)^\frac{3}{4}$\unsure{
        common words have larger probabilities to be sampled, while rare words have smaller ones.
    }
    \begin{align}
        p(w_{t+j}|w_t) & \approx p(D=1|w_t,w_{t+j})\prod_{k\in\mathcal{K}} p(D=0|w_t,w_k) \\
        p(D=1|w_t,w_{t+j}) & = \sigma(\bm{v}_{w_{t+j}}^\mathsf{T}\bm{v}_{w_t}) \\
        p(D=0|w_t,w_k) & = 1 - \sigma(\bm{v}_{w_k}^\mathsf{T}\bm{v}_{w_t})
    \end{align}
\end{itemize}

